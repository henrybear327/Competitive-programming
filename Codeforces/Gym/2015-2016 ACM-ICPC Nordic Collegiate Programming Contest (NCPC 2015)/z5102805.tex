\documentclass[10pt]{article}
\usepackage[utf8]{inputenc}
\usepackage[a4paper]{geometry}

\usepackage{pdfpages}
\usepackage[colorlinks=true, urlcolor=cyan]{hyperref}

\usepackage{listings}
\usepackage{xcolor}

\definecolor{mygreen}{rgb}{0,0.6,0}
\definecolor{mygray}{rgb}{0.5,0.5,0.5}
\definecolor{mymauve}{rgb}{0.58,0,0.82}

\lstset{ %
  language=C++,								  % the language of the code
  backgroundcolor=\color{black!5}, % set backgroundcolor,   % choose the background color; you must add \usepackage{color} or \usepackage{xcolor}
  basicstyle=\footnotesize,        % the size of the fonts that are used for the code
  breakatwhitespace=false,         % sets if automatic breaks should only happen at whitespace
  breaklines=true,                 % sets automatic line breaking
  captionpos=b,                    % sets the caption-position to bottom
  commentstyle=\color{mygreen},    % comment style
  deletekeywords={...},            % if you want to delete keywords from the given language
  escapeinside={\%*}{*)},          % if you want to add LaTeX within your code
  extendedchars=true,              % lets you use non-ASCII characters; for 8-bits encodings only, does not work with UTF-8
  frame=single,	                   % adds a frame around the code
  keepspaces=true,                 % keeps spaces in text, useful for keeping indentation of code (possibly needs columns=flexible)
  keywordstyle=\color{blue},       % keyword style
  otherkeywords={*,...},           % if you want to add more keywords to the set  
  numbers=left,                    % where to put the line-numbers; possible values are (none, left, right)
  numbersep=7pt,                   % how far the line-numbers are from the code
  numberstyle=\tiny\color{mygray}, % the style that is used for the line-numbers
  stepnumber=1,                    % the step between two line-numbers. If it's 1, each line will be numbered
  rulecolor=\color{black},         % if not set, the frame-color may be changed on line-breaks within not-black text (e.g. comments (green here))
  showspaces=false,                % show spaces everywhere adding particular underscores; it overrides 'showstringspaces'
  showstringspaces=false,          % underline spaces within strings only
  showtabs=false,                  % show tabs within strings adding particular underscores
  stringstyle=\color{mymauve},     % string literal style
  tabsize=4,	                   % sets default tabsize to 4 spaces
  title=\lstname                   % show the filename of files included with \lstinputlisting; also try caption instead of title
}

\lstdefinestyle{customc}{
  belowcaptionskip=1\baselineskip,
  breaklines=true,
  frame=L,
  xleftmargin=\parindent,
  language=C++,
  showstringspaces=false,
  basicstyle=\footnotesize\ttfamily,
  keywordstyle=\bfseries\color{green!40!black},
  commentstyle=\itshape\color{purple!40!black},
  identifierstyle=\color{blue},
  stringstyle=\color{orange},
}

\lstset{escapechar=@,style=customc}

\title{2015-2016 ACM-ICPC Nordic Collegiate Programming Contest (NCPC 2015)} % // is line break
\author{Chun-Hung Tseng}
\date{\today}

% Use this tag to surround the entire document, a must
\begin{document} 

\begin{titlepage}
\maketitle
\end{titlepage}

%By default, the paragraphs are indented by 1.5 times the point size of the current font. Also, there is no extra blank space inserted between the paragraphs. In the sections below is described how to change that.

%A
\includepdf[pages={3,4}]{problem_statement.pdf}

\section*{Problem A}

This is a very interesting problem, which I got 12 WA during virtual participation on Codeforces. Draw the diagram out, you will realize that this problem is about tree diameter and radii pretty quickly. \\

You are given a forest with n vertices to start with, and you have to add one edge for each tree in the forest to connect all of them together eventually, such that the forest will become a single tree, where the furthest two leaves are of the shortest distance.\\ 

Finding the diameter and radii is the part that got me 12 WA, because I didn't implement it the right way for the first 10+ attempts. Here is a great article on \href{http://codeforces.com/blog/entry/17974}{Codeforces} that have great explanations on this. \\

Let's denote $D$ for the largest diameter of all trees in the forest, $A$ for the largest radius, $B$ for the second-largest radius(if exists), and $C$ for the third-largest radius of all trees(if exists). The answer will be $max(D, A+B+1, B+C+2)$. We get this formula by two crucial observations: connect to trees using their center(s) only, and always connect to the center of the largest diameter. In this way, we can minimize the distance between two furthest leaves.\\

The solution is $O(n)$, because finding the diameter and radii are all linear time algorithms.

\lstinputlisting[language=c++]{A/main.cpp}

\newpage

%C
\includepdf[pages={7}]{problem_statement.pdf}

\section*{Problem C}

The easiest problem in this problem set.\\

Find the total differences between the input string and string "PERPERPER..." ("PER" repeats forever), where the input string is given in a length of a multiple of 3.\\

The solution is $O(n)$, just linear scan and check the differences.\\

\lstinputlisting[language=c++]{C/main.cpp}

\newpage

%D
\includepdf[pages={9}]{problem_statement.pdf}

\section*{Problem D}

Finding the minimal number of server required to process all requests is equal to finding the maximal request at a specific time. \\

We can tackle the problem by splitting each request into two parts: when the request is received, we create a tuple (time, 1); when the request is finished, we create a tuple (time + 1000, 0). Sort all tuples by time (non-decreasing order) and status (non-decreasing order). Iterate over all tuples, and keep track of the maximal sum at a specific moment. \\

The maximal sum will tell us what is the maximal number of requests being processed at a specific moment. Divide maximal sum by the number of requests a server can handle per second, take the ceiling of the result, and will get the answer.\\

The solution is $O(nlog(n))$ because sorting is involved.

\lstinputlisting[language=c++]{D/main.cpp}

\newpage

%E
\includepdf[pages={11}]{problem_statement.pdf}

\section*{Problem E}

We are trying to record as many TV shows as we can, so we first sort the TV shows by end time, in non-decreasing order, and then we try to add them to the tracks greedily.\\

Add the TV shows to the track using the following principle: extend the track that has the end time closest to your start time. \\

Because the data structure that we used for maintaining the data is c++ multiset, the solution is $O(nlog(k))$, where n is the total number of TV shows and k is the total tracks we have. 

\lstinputlisting[language=c++]{E/main.cpp}

\newpage

%G
\includepdf[pages={15,16}]{problem_statement.pdf}

\section*{Problem G}

\lstinputlisting[language=c++]{G/main.cpp}

\newpage

\end{document}
